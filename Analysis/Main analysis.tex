\documentclass[a4paper,12pt]{article}

\usepackage[latin1]{inputenc}
\usepackage[spanish,es-tabla]{babel}
\usepackage{amssymb}
\usepackage{amsmath}
\usepackage{latexsym}
\usepackage{graphicx}
\usepackage{enumerate}
\usepackage{setspace}
\usepackage{units}
\usepackage{color}
\usepackage{float}
\usepackage{longtable}
\usepackage{mathrsfs}
\usepackage{tikz}
\usepackage{pgfplots}
\usepackage{relsize}
%\usepackage{mathabx}
\usepackage{ulem}

\bibliographystyle{alpha}

%\doublespacing
\clubpenalty=10000
\widowpenalty=10000

\DeclareMathOperator{\arccot}{arc\ cot}
\DeclareMathOperator{\arcsec}{arc\ sec}
\DeclareMathOperator{\arccsc}{arc\ csc}
\DeclareMathOperator{\sech}{sech}
\DeclareMathOperator{\csch}{csch}
\DeclareMathOperator{\argsenh}{arg\ senh}
\DeclareMathOperator{\argcosh}{arg\ cosh}
\DeclareMathOperator{\argtgh}{arg\ tgh}
\DeclareMathOperator{\argcoth}{arg\ coth}
\DeclareMathOperator{\argsech}{arg\ sech}
\DeclareMathOperator{\argcsch}{arg\ csch}
\DeclareMathOperator{\im}{im}
\DeclareMathOperator{\mcd}{mcd}
\DeclareMathOperator{\modX}{m\acute od}
\DeclareMathOperator{\Cov}{Cov}

\pgfplotsset{compat=1.12}

\begin{document}

\newlength{\origpar}
\setlength{\origpar}{\parindent}
\setlength{\parindent}{0pt}
\setlength{\parindent}{\origpar}

\renewcommand {\baselinestretch} {1.0}
\renewcommand*{\arraystretch}{1.5}

\newcommand{\paren}[1]{\ensuremath{\left(#1\right)}}
\newcommand{\corch}[1]{\ensuremath{\left[#1\right]}}
\newcommand{\llave}[1]{\ensuremath{\left\{#1\right\}}}
\newcommand{\angparen}[1]{\ensuremath{\left<#1\right>}}
\newcommand{\abs}[1]{\ensuremath{\left|#1\right|}}
\newcommand{\vecnorm}[1]{\ensuremath{\left|\left|#1\right|\right|}}
\newcommand{\pdiff}[2]{\ensuremath{\dfrac{\partial#1}{\partial#2}}}
\newcommand{\tlap}[1]{\ensuremath{\mathscr L\llave{#1}}}
\newcommand{\itlap}[1]{\ensuremath{\mathscr L^{-1}\llave{#1}}}
\newcommand{\Section}[1]{\renewcommand{\thesection}{#1}\section}
\newcommand{\floor}[1]{\ensuremath{\left\lfloor#1\right\rfloor}}
\newcommand{\ceil}[1]{\ensuremath{\left\lceil#1\right\rceil}}

\newcommand{\zm}[1]{\ensuremath{\mathbb Z/#1\mathbb Z}}

\definecolor{grey}{rgb}{0.50,0.50,0.50}
\definecolor{moradoesfruta}{rgb}{1,0,1}
\definecolor{orangeisthenewmorado}{rgb}{1,0.5,0}

\newcommand{\inRed}[1]{\textcolor{red}{#1}}
\newcommand{\inBlue}[1]{\textcolor{blue}{#1}}
\newcommand{\inGreen}[1]{\textcolor{green}{#1}}
\newcommand{\inPurple}[1]{\textcolor{moradoesfruta}{#1}}
\newcommand{\inOrange}[1]{\textcolor{orangeisthenewmorado}{#1}}

\newcommand{\SectionR}[1]{\renewcommand{\thesection}{\inRed{#1}}\section}
\newcommand{\SectionG}[1]{\renewcommand{\thesection}{\inGreen{#1}}\section}
\newcommand{\SectionO}[1]{\renewcommand{\thesection}{\inOrange{#1}}\section}

\newcommand{\matlab}[1]{\begin{singlespace}\noindent\footnotesize{\texttt{\textcolor{black}{#1}}}\end{singlespace}}
\newcommand{\maxima}[1]{\begin{singlespace}\noindent\footnotesize{\texttt{\textcolor{blue}{#1}}}\end{singlespace}}
\newcommand{\maximaoutput}[1]{\begin{singlespace}\noindent\footnotesize{\texttt{\textcolor{grey}{#1}}}\end{singlespace}}
\newcommand{\scilab}[1]{\begin{singlespace}\noindent\footnotesize{\texttt{\textcolor{black}{#1}}}\end{singlespace}}

% Magia negra de Latex a continuaci�n (copiada de http://tex.stackexchange.com/questions/44235/is-there-a-way-to-do-an-upside-down-widehat).
\makeatletter
\DeclareRobustCommand\widecheck[1]{{\mathpalette\@widecheck{#1}}}
\def\@widecheck#1#2{%
    \setbox\z@\hbox{\m@th$#1#2$}%
    \setbox\tw@\hbox{\m@th$#1%
       \widehat{%
          \vrule\@width\z@\@height\ht\z@
          \vrule\@height\z@\@width\wd\z@}$}%
    \dp\tw@-\ht\z@
    \@tempdima\ht\z@ \advance\@tempdima2\ht\tw@ \divide\@tempdima\thr@@
    \setbox\tw@\hbox{%
       \raise\@tempdima\hbox{\scalebox{1}[-1]{\lower\@tempdima\box
\tw@}}}%
    {\ooalign{\box\tw@ \cr \box\z@}}}
\makeatother
% Fin de la magia negra.

\setcounter{secnumdepth}{3}
\section{Multiples of 3 and 5}
\textbf{Solution}: 233168. \textit{Solved: Sat, 8 Aug 2015, 08:09}.

\textbf{Math knowledge used}: triangular numbers.

\textbf{Programming techniques used}: none.

It's very simple using inclusion-exclusion. We count how many multiples of $3$, $5$ and $15$ are there, then we use triangular numbers to determine each sum, then we return $\mathrm{sum}\paren3+\mathrm{sum}\paren5-\mathrm{sum}\paren{15}$.

\section{Even Fibonacci numbers}
\textbf{Solution}: 4613732. \textit{Solved: Sat, 8 Aug 2015, 08:23}.

\textbf{Math knowledge used}: recurrence equation theory.

\textbf{Programming techniques used}: none.

You can bruteforce this very easily. Or you can use recurrence equations! Which is of course my preferred method since it's more mathematical and, much more importantly, more scalable.

So, fibonacci numbers follow this formula:
\begin{equation*}
f\paren n=\dfrac{\varphi^n-\paren{1-\varphi}^n}{\sqrt5}.
\end{equation*}

And even fibonacci numbers are precisely those for which $n=\dot3$. That is, the numbers we are looking for follow this other formula:
\begin{equation*}
f_{\mathrm even}\paren n=\dfrac{\varphi^{3n}-\paren{1-\varphi}^{3n}}{\sqrt5}.
\end{equation*}

From this, considering that $\paren{1-\varphi}^{3n}$ tends to $0$ very rapidly, and calling $L=4e6$ the imposed limit, we can get a suitable formula for the largest $n$ we need:
\begin{equation*}
N=\floor{\dfrac{\log L}{3\log\varphi}}.
\end{equation*}

Finally, since we need the sum, we use a basic summation formula (again, considering that the $\paren{1-\varphi}^{3n}$ terms tend quickly to zero):
\begin{equation*}
\sum_{i=0}^n\dfrac{\varphi^{3n}-\paren{1-\varphi}^{3n}}{\sqrt5}\approx\sum_{i=0}^n\dfrac{\varphi^{3n}}{\sqrt5}=\dfrac{\varphi^{3n+3}-\varphi^3}{\paren{\varphi^3-1}\sqrt5}.
\end{equation*}
Rounding this value into the nearest integer we get the solution.

\section{Largest prime factor}
\textbf{Solution}: 6857. \textit{Solved: Sat, 8 Aug 2015, 08:33}.

\textbf{Math knowledge used}: Erathostenes sieve.

\textbf{Programming techniques used}: none.

Very simple, just enumerate all the primes up to $\sqrt L$ (with $L$ being the limit) and then divide until either we hit $1$ (meaning that the last prime we divided by is the result) or we run out of primes (meaning that the remaining number is the last prime we're looking for).

There are many methods that can be used to iterate over the primes, but the Erathostenes sieve is good enough. Saying that it will be used profusely is a big, big understatement.

\section{Largest palindrome product}
\textbf{Solution}: 906609. \textit{Solved: Sat, 8 Aug 2015, 08:39}.

\textbf{Math knowledge used}: modular arithmetic.

\textbf{Programming techniques used}: none.

The ``modular arithmetic'' I'm referring to is just the fact that I know that all palindrome numbers with an even amount of digits are multiples of $11$. So, the method to found this answer is as simple as iterating downwards over a pair of factors, but this bit of knowledge allows me to reduce the search space by 11, because one of the factors iterates over just multiples of $11$.

\section{Smallest multiple}
\textbf{Solution}: 232792560. \textit{Solved: Sat, 8 Aug 2015, 08:49}.

\textbf{Math knowledge used}: none.

\textbf{Programming techniques used}: none.

I'm not sure if ``none'' qualifies as the list of special knowledge used. I just factored all the numbers and then calculated the least common multiple in a more or less intelligent way (i.e. I have the factors, so no need to use Euclid's algorithm). This is the first time I used my DivisorHolder class, which is used for a fuckton of problems all over the list.

\section{Sum square difference}
\textbf{Solution}: 25164150. \textit{Solved: Sat, 8 Aug 2015, 08:54}.

\textbf{Math knowledge used}: basic Faulhaber formulas.

\textbf{Programming techniques used}: none.

The sum of all natural numbers up to $n$ is $\dfrac{n\paren{n+1}}2$, and if we square this we get:
\begin{equation*}
f\paren n=\dfrac{n^4+2n^3+n^2}4.
\end{equation*}

On the other hand, the sum of the squares of all the natural numbers up to $n$ is
\begin{equation*}
g\paren n=\dfrac{2n^3+3n^2+n}6.
\end{equation*}

Subtracting both numbers we get the formula we need:
\begin{equation*}
f\paren n-g\paren n=\dfrac{3n^4+2n^3-3n^2-2n}{12}.
\end{equation*}

Plugging in $n=100$ we get the result. By the way, it can be easily proven that despite the denominator this number is always an integer.



\end{document}
