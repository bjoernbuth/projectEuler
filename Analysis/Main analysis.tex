\documentclass[a4paper,12pt]{article}

\usepackage[latin1]{inputenc}
\usepackage[spanish,es-tabla]{babel}
\usepackage{amssymb}
\usepackage{amsmath}
\usepackage{latexsym}
\usepackage{graphicx}
\usepackage{enumerate}
\usepackage{setspace}
\usepackage{units}
\usepackage{color}
\usepackage{float}
\usepackage{longtable}
\usepackage{mathrsfs}
\usepackage{tikz}
\usepackage{pgfplots}
\usepackage{relsize}
\usepackage{mathabx}
\usepackage{ulem}

\bibliographystyle{alpha}

%\doublespacing
\clubpenalty=10000
\widowpenalty=10000

\DeclareMathOperator{\arccot}{arc\ cot}
\DeclareMathOperator{\arcsec}{arc\ sec}
\DeclareMathOperator{\arccsc}{arc\ csc}
\DeclareMathOperator{\sech}{sech}
\DeclareMathOperator{\csch}{csch}
\DeclareMathOperator{\argsenh}{arg\ senh}
\DeclareMathOperator{\argcosh}{arg\ cosh}
\DeclareMathOperator{\argtgh}{arg\ tgh}
\DeclareMathOperator{\argcoth}{arg\ coth}
\DeclareMathOperator{\argsech}{arg\ sech}
\DeclareMathOperator{\argcsch}{arg\ csch}
\DeclareMathOperator{\im}{im}
\DeclareMathOperator{\mcd}{mcd}
\DeclareMathOperator{\modX}{m\acute od}
\DeclareMathOperator{\Cov}{Cov}

\pgfplotsset{compat=1.12}

\begin{document}

\newlength{\origpar}
\setlength{\origpar}{\parindent}
\setlength{\parindent}{0pt}
\setlength{\parindent}{\origpar}

\renewcommand {\baselinestretch} {1.0}
\renewcommand*{\arraystretch}{1.5}

\newcommand{\paren}[1]{\ensuremath{\left(#1\right)}}
\newcommand{\corch}[1]{\ensuremath{\left[#1\right]}}
\newcommand{\llave}[1]{\ensuremath{\left\{#1\right\}}}
\newcommand{\angparen}[1]{\ensuremath{\left<#1\right>}}
\newcommand{\abs}[1]{\ensuremath{\left|#1\right|}}
\newcommand{\vecnorm}[1]{\ensuremath{\left|\left|#1\right|\right|}}
\newcommand{\pdiff}[2]{\ensuremath{\dfrac{\partial#1}{\partial#2}}}
\newcommand{\tlap}[1]{\ensuremath{\mathscr L\llave{#1}}}
\newcommand{\itlap}[1]{\ensuremath{\mathscr L^{-1}\llave{#1}}}
\newcommand{\Section}[1]{\renewcommand{\thesection}{#1}\section}
\newcommand{\floor}[1]{\ensuremath{\left\lfloor#1\right\rfloor}}
\newcommand{\ceil}[1]{\ensuremath{\left\lceil#1\right\rceil}}

\newcommand{\zm}[1]{\ensuremath{\mathbb Z/#1\mathbb Z}}

\definecolor{grey}{rgb}{0.50,0.50,0.50}
\definecolor{moradoesfruta}{rgb}{1,0,1}
\definecolor{orangeisthenewmorado}{rgb}{1,0.5,0}

\newcommand{\inRed}[1]{\textcolor{red}{#1}}
\newcommand{\inBlue}[1]{\textcolor{blue}{#1}}
\newcommand{\inGreen}[1]{\textcolor{green}{#1}}
\newcommand{\inPurple}[1]{\textcolor{moradoesfruta}{#1}}
\newcommand{\inOrange}[1]{\textcolor{orangeisthenewmorado}{#1}}

\newcommand{\SectionR}[1]{\renewcommand{\thesection}{\inRed{#1}}\section}
\newcommand{\SectionG}[1]{\renewcommand{\thesection}{\inGreen{#1}}\section}
\newcommand{\SectionO}[1]{\renewcommand{\thesection}{\inOrange{#1}}\section}

\newcommand{\matlab}[1]{\begin{singlespace}\noindent\footnotesize{\texttt{\textcolor{black}{#1}}}\end{singlespace}}
\newcommand{\maxima}[1]{\begin{singlespace}\noindent\footnotesize{\texttt{\textcolor{blue}{#1}}}\end{singlespace}}
\newcommand{\maximaoutput}[1]{\begin{singlespace}\noindent\footnotesize{\texttt{\textcolor{grey}{#1}}}\end{singlespace}}
\newcommand{\scilab}[1]{\begin{singlespace}\noindent\footnotesize{\texttt{\textcolor{black}{#1}}}\end{singlespace}}

% Magia negra de Latex a continuaci�n (copiada de http://tex.stackexchange.com/questions/44235/is-there-a-way-to-do-an-upside-down-widehat).
\makeatletter
\DeclareRobustCommand\widecheck[1]{{\mathpalette\@widecheck{#1}}}
\def\@widecheck#1#2{%
    \setbox\z@\hbox{\m@th$#1#2$}%
    \setbox\tw@\hbox{\m@th$#1%
       \widehat{%
          \vrule\@width\z@\@height\ht\z@
          \vrule\@height\z@\@width\wd\z@}$}%
    \dp\tw@-\ht\z@
    \@tempdima\ht\z@ \advance\@tempdima2\ht\tw@ \divide\@tempdima\thr@@
    \setbox\tw@\hbox{%
       \raise\@tempdima\hbox{\scalebox{1}[-1]{\lower\@tempdima\box
\tw@}}}%
    {\ooalign{\box\tw@ \cr \box\z@}}}
\makeatother
% Fin de la magia negra.

\setcounter{secnumdepth}{3}
\section{Multiples of 3 and 5}
\textbf{Solution}: 233168.

\textbf{Math knowledge used}: triangular numbers.

\textbf{Programming techniques used}: none.

It's very simple using inclusion-exclusion. We count how many multiples of $3$, $5$ and $15$ are there, then we use triangular numbers to determine each sum, then we return $\mathrm{sum}\paren3+\mathrm{sum}\paren5-\mathrm{sum}\paren{15}$.


\end{document}
